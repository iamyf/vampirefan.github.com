\documentclass[11pt,twocolumn]{ctexart}
\usepackage[top=1in,bottom=1in,left=1in,right=1in]{geometry}
\usepackage[bf,small,pagestyles]{titlesec}
\usepackage[dvips]{graphicx}
\usepackage{float}
\usepackage{indentfirst}
\usepackage[ps2pdf=true,colorlinks]{hyperref}
\usepackage[figure,table]{hypcap} % Correct a problem with hyperref
\hypersetup{
   bookmarksnumbered,
   pdfstartview={FitH},
   citecolor={black},
   linkcolor={black},
   urlcolor={black},
   pdfpagemode={UseOutlines}
}
\usepackage{multicol}
\def\pgfsysdriver{pgfsys-dvipdfmx.def}
\pagestyle{plain}

\begin{document}
\title{\textbf{光传输网络中的挑战}\\[1ex]路由和波长分配问题}
\author{王帆 5070309862 F0703034}
\date{\today}

\twocolumn[
\begin{@twocolumnfalse}
\phantomsection
\addcontentsline{toc}{section}{Title}
\maketitle

\phantomsection
\addcontentsline{toc}{section}{Abstract}
\begin{abstract}
光通信作为一门新兴技术,其近年来发展速度之快、应用面之广是通信史上罕见的,也是世界新技术革命的重要标志和未来信息社会中各种信息的主要传送工具。光纤通信的问世使高速率、大容量的通信成为可能,目前它已成为最主要的信息传输技术。所谓光纤通信,是指利用光导纤维传输光波信号的通信方式。本文首先对光通信领域中所遇到的挑战做简单的综述,然后就其中的路由和波长分配问题作较为深入的叙述,最后提出一些自己对这一问题的看法。
\end{abstract}
\end{@twocolumnfalse}
]
%\begin{multicols}{2}
\section{引言}
本文作为光纤通信的大作业,题目为我眼中的光通信系统,就现今光通信技术中的问题展开讨论,并找出一定的可实行方案进行叙述。本文的结构安排如下:在第二章对光通信领域中遇到的几个主要挑战做了比较简单的综述,第三章则针对路由和波长分配问题作比较深入的叙述,其中主要是对几种路由和波长分配算法的描述。\cite{1,6,8,9}最后是自己对这些问题的一些看法,以及做这个大作业的感想。

\section{综述:光网络传输中的挑战}
光网络的发展非常之迅速,从最开始的不透明架构(opaque)到半透明结构(translucent),到现今的全光网透明架构(all-optical, transparent)。在半透明的光网络中,电再生技术(e.g. 光-电-光 转换技术)的应用还不是很广泛。到全光网络中,路由和波长分配(Routing and Wavelength Assignment, RWA)问题中的物理损伤是现在很多研究团体的重点。\cite{1,6,8,9}此外,随着波分复用技术(WDM, Wavelength Division Multiplexing)的快速提高,光网络的传输容量也是与日俱增。与之产生的问题也随之而来:任何情况的网络故障都会造成巨大的数据丢失和传输阻塞。因此,光网络的生存性(Survivability)研究也成为光通信中的热门话题。\cite{2}全光网的建设中,为了使更多的用户能够宽带接入城域网,长距离光纤接入(Long-Reach optical access)技术也是急待提高的领域。\cite{3,4}光传输中,交换速度(switching speed)是另一个重要的话题。光交换与传输技术也面临着诸多的挑战。\cite{5,7}

\subsection{路由和波长分配(RWA)问题}
正如上述,光网络在近年来一直向全光网透明架构演化。透明全光网络中,光信号从源节点穿过光路到达目的节点的全部过程都是处于光域中的,不需要光-电-光转换(optical-electronic-optical conversion)。因没有电再生过程(regeneration),在光传输过程中物理损伤(physical impairments)不断地累积,造成了信号的失真以及噪声,其中包括线性损伤(e.g. 光器件干扰等)及更复杂的非线性损伤。当网络中物理损伤严重时,极大影响了服务质量(QoS,Quality of Service)。损伤原因的多样与复杂挑战光网络路由选择和波长分配(RWA)问题。RWA问题已经有很多的算法与研究。

\subsection{光网络的生存性问题}
网络生存性是指在节点或链路故障情况下,网络能够保护受故障影响的业务量,继续维持其正常通信的能力。而如今光网络的广泛应用使其生存性的研究变得至关重要。

宽带网络中的核心IP业务直接承载于WDM系统的IP over WDM光网络随着IP业务的增长及WDM的快速提高而日渐成为主流的网络结构。这种IP over WDM光网络的生存性研究可以在\cite{10}中更详细的了解。

多域网络是由几个单域网络,相互内部连接起来的网络。每个单域网络可以被视为一个独立的网络,有其经营和管理自己的规范,以提供服务。由于一些扩展性的限制一个域的内部拓扑结构并不对外共享。应此在多域网络的节点中没有多域网络的完整信息。这样多域网络保护更加困难。正因为如此,多域网络的生存性也是极为重要的。\cite{2}

\subsection{长距离光纤接入问题}
随着光通信技术的进步,为了使更多的用户能够宽带接入城域网,长距离光纤接入技术也是急待提高的领域。在长距离光纤接入的问题上,光网络的各个层面,包括物理层的光学元件的控制和上层的管理问题都是必须考虑的。Huan Song的两篇论文\cite{3,4}对这些部分做了比较详细的综述。

\subsection{光交换与传输问题}
由于当今数据传输的需求量是日益增大,单单电子传输是不可能完成的。因此光传输数据的需求是刻不容缓。在光传输系统中,每根光纤可以传输10到100种波长的光信号,而每种波长的光信号能调制至10Gps甚至更高。而在光传输成为可能之前,电信号在现今被认为是一种成熟的技术,但随着日渐增大的需求量,光传输是可行的解决方案,并极大程度上依靠光交换与传输技术。

在光交换技术中,很多人都提出了各种模型,主要有 Opto-mechanical switches, Electro-optic switches,  Micro electromechanical system devices, Semiconductor optical amplifier switch等。在\cite{5,7}有比较详细的叙述。

\section{路由和波长分配(RWA)}
光路径的定义是用于连接客户的光通路,即光通道源节点与目的节点之间的通路。一条光路径可以使用该通路的一个或几个波长。

路由是控制面的一种功能,用于选择路径和建立连接,此连接往往穿越一个或几个传送网。

在光域中,将路由过程称为RWA问题,为了选择合适的光路径来满足流量工程要求,就必须找到光路径中物理节点和链路(路由子问题),同时找到该光路径链路上的一个或几个波长(波长分配子问题),才能优化网络资源。

\subsection{基于损伤感知的动态RWA (Impairment-aware RWA, IRWA)}
光网络构架的演进是从传统的不透明到全光网透明架构。图~\ref{EVO}转自\cite{1}展示了其光网络的演进.
\begin{figure}[!hbtp]
  \begin{center}
  \includegraphics[width=0.5\textwidth]{EVO}
  \end{center}
  \caption{光通信网络的演进}
  \label{EVO}
\end{figure}

光信号的传输要通过在每个光交换或路由处的OEO转换。由于时机和经济的考虑,光信号的传输距离是有限的(e.g. 2000-2500km)。为了超越这个极限,信号的再生(re-amplify, re-shape and re-time, 称为3R)是至关重要的。然而在光信号的传输过程中,如果能更有效地降低其物理损伤,光信号的传输效率也能明显地提高,同时信号的再生所需的工作也能从一定程度上减轻。

物理层的损伤可以被看成是对路由和波长分配决定的一种约束 (i.e. physical layer impairment constrained (PLIC-RWA));或是有路由和波长分配来考虑之(i.e. physical layer impairment aware (PLIA-RWA))。在后者中,找到合适物理损伤的路由决定是可能的。简单地说既考虑物理损伤的路由和波长分配决定(PLI-RWA)。其算法可以简单归纳如下:

首先利用经典RWA算法从可用资源中选路,其次是计算该波长路径上的物理损伤,验证该光路上的信号至目的节点时其OSNR值(光信噪比, Optical Signal Noise Rate)能否满足最低需求,如能满足需求则使用该光路建立连接,否则换下一条光路验证其物理损伤情况。图~\ref{IRWA}转自\cite{8}显示了其算法的基本步骤.
\begin{figure}[!hbtp]
  \begin{center}
  \includegraphics[width=0.5\textwidth]{IRWA}
  \end{center}
  \caption{具有损伤感知能力的动态RWA算法}
  \label{IRWA}
\end{figure}

\subsection{基于网络编码的光组播树优化RWA}
光层组播是指通过在物理层对组播数据进行复制, 并使用光分波器/合波器等无源器件实现点到多点的连接。在WDM网络中实现光层组播首先要建立点到多点的光连接方式, 从而形成光组播树, 并为其分配波长, 即RWA。周迎富等人利用分层图模型提出光组播树多路径波长优化算法。图~\ref{MT}转自\cite{8}显示了其模型.
\begin{figure}[!hbtp]
  \begin{center}
  \includegraphics[width=0.5\textwidth]{MT}
  \end{center}
  \caption{基于网络编码的光组播树优化RWA模型}
  \label{MT}
\end{figure}

\subsection{基于蜜蜂群优化算法的RWA}
Goran Z. and Teodorovic D 等人提出了基于蜜蜂群优化算法(Bee Colony Optimization, BCO)的RWA(BCO-RWA)。\cite{6} BCO-RWA算法从某种程度上看是一种路由拓扑结构,通过对这种拓扑结构的阐释和理解,对光传输的路由进行选择,从而优化RWA的问题。

模型通过模拟蜜蜂群选择路线来确定路由的最佳路线,其模拟的网络由即RWA。周迎富等人利用分层图模型提出光组播树多路径波长优化算法。图~\ref{BCO}转自\cite{6}显示。
\begin{figure}[!hbtp]
  \begin{center}
  \includegraphics[width=0.5\textwidth]{BCO}
  \end{center}
  \caption{基于蜜蜂群算法的虚拟网络}
  \label{BCO}
\end{figure}

\section{基于无线传感器路由的想法}
通过对第三章中几种典型的优化RWA算法,比较无线传感器中的一些路由协议,我觉得对于在光通信中的RWA虽然是一个NP问题,但其实质上是一个优化路由问题。其目的并不是要完全的解决它,而是尽一切可能的去优化,使之能够达到应用层面的需求。

对于第一种IRWA,可以对比到无线传感器中的能量认知路由(Energy-aware routing)。在无线传感器网络中,能源的约束是决定网络生命期的最主要因素。因此对于节点间的通信,以及节点与基站的通信都应该以尽可能减少能量开销为考虑的首要因素。针对此点,很多人也建立了各种各样的模型来达到此目的。比如,在通信前选好路径、将不活动的节点处于休眠状态、设置数据门限(当数据变化量大于某值才发送信号)等一系列的模型使各种不同的路由方式能适应各种不同的网络需要。我认为IRWA也能借鉴这些模型中的一部分来完成光通信网络的需要。

对于第二种基于网络编码的光组播树优化RWA模型,无线传感器网络中也有对应的路由模型。等级路由(hierarchical routing)就是一种典型的基于簇状的路由方式。其模型将网络节点分为各个等级,最简单的是挑出一部分作为一簇节点中的“簇头”,这样,数据线从各节点传出,先经过簇头的封装优化,再传送给基站;同样,如果基站需要的数据在哪一个簇,它可以直接和簇头联系,而避免了广播需求而消耗的大量能量。图~\ref{TEEN}转自\cite{11}显示了一种典型的无线传感器网络等级路由协议。
\begin{figure}[!hbtp]
  \begin{center}
  \includegraphics[width=0.5\textwidth]{TEEN}
  \end{center}
  \caption{TEEN和APTEEN中的等级簇}
  \label{TEEN}
\end{figure}

而对于基于蜜蜂群优化算法的RWA,是一种比较特殊的优化算法,Goran Z.等人在将这种算法运用到RWA问题上时做了大量的计算仿真以及比较\cite{6}。正如文中所说,这种算法具有一定的可行性,并且其性能也很优秀于很多其他模型,但是由于其复杂程度以及不确定性等因素,只能适用于相应需求的光网络系统。

事实上,由于光网络通信中有各种各样的需求与条件,不论是哪一种模型,都是基于某种或某些条件而建立的,并不能完全的适合各种网络需求。这样我们只能尽可能的利用不同的模型方式来满足各种不同的网络需求。



\phantomsection 
\addcontentsline{toc}{section}{References}
\begin{thebibliography}{99}
\bibitem{1} Siamak Azodolmolky et al.,
	{\it A survey on physical layer impairments aware routing and wavelength assignment algorithms in optical networks}
	Computer Networks, Dec. 2008
	
\bibitem{2} Hamza Drid, Bernard Cousin, Miklos Molnar, Samer Lahoud,
	{\it A Survey of Survivability in Multi-Domain Optical Networks}
	Computer Communications, 2010 
	
\bibitem{3} Huan Song, Byoung-Whi Kim, and Biswanath Mukherjee 
	{\it Long-Reach Optical Access Networks: A Survey of Research Challenges, Demonstrations, and Bandwidth Assignment Mechanisms}
	 IEEE Communications Survey and Tutorials, accepted, 2009

\bibitem{4} Huan Song et al.,
	{\it Long-Reach Passive Optical Networks}
	cs.ucdavis.edu, 2009
	
\bibitem{5} Ravinder Yadav, Rinkle Rani Aggarwal,
	{\it Survey and Comparison of Optical Switch Fabrication Techniques and Architectures}
	Department of Computer Science and Engineering, Thapar University, Patiala, 2010
	
\bibitem{6} G.Z., Teodorovic~D., Acimovic-Raspopovic V.S.,
	{\it Routing and Wavelength Assignment in All-Optical Networks Based on the Bee Colony Optimization}
	AI Communications - The European Journal on Artificial Intelligence, (in press).
	 
\bibitem{7} Martin Maier, Martin Reisslein,
	{\it Trends in Optical Switching Techniques: A Short Survey}
	IEEE Network, 2008

\bibitem{8} 赵继军, 王丽荣, 纪越峰, 徐大雄,
	{\it 基于损伤感知的动态RWA算法性能比较研究}
	电子与信息学报, Mar. 2010

\bibitem{9} 周迎富, 阳小龙,
	{\it 基于网络编码的光组播树优化RWA研究}
	计算机应用研究, Nov. 2009

\bibitem{10} 李景聪, 吴德明, 徐安士
	{\it IP over WDM光网络及其生存性问题讨论}
	光电子$\cdot$激光 Feb. 2002
	
\bibitem{11} Kemal Akkaya, Mohamed Younis 
	{\it A survey on routing protocols for wireless sensor networks}
	Nov.~2003
	
\end{thebibliography}
%\end{multicols}
\end{document}
