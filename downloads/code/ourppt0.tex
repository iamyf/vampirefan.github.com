\documentclass[xcolor=dvipsnames]{beamer}
\usecolortheme[named=Black]{structure}
\usetheme[height=7mm]{Singapore}
\usepackage{amsmath}

\begin{document}
\setbeamertemplate{background}{\includegraphics[width=\paperwidth]{sjtu1.png}}
\setbeamertemplate{navigation symbols}{}%取消导航条
\setbeamertemplate{items}[ball] 
\setbeamertemplate{caption}[numbered]  

\title{\textsc{Routing and Topology Control in Wireless Sensor Networks}\\[2ex]\begin{large} \textbf{By Group Three}\end{large}}
\author{
\begin{tabular}{llll}
Wang Haosen & Wang Fan & Liu Wuchong & Luo Bing\\
5070309861 & 5070309862 & 5070309857 & 5070309858
\end{tabular}}
\date{\today}

\begin{frame}
\titlepage
\end{frame}

\section{Outline}
\begin{frame}
\frametitle{Outline}
\begin{itemize}
\item<1-> Introduction to WSN
\item<2-> Overview of WSN Routing
\item<3-> OTC in WSNs
\item<4-> Random Walks on Digraphs
\end{itemize}
\end{frame}


\section{Introduction to WSN}
\begin{frame}
\frametitle{Introduction to the sensor network}
\begin{itemize}
\item<1-> A sensor network is composed of a large number of sensor nodes that are densely deployed either inside the phenomenon or very close to it.
\item<2-> Realization of these and other sensor network applications require wireless ad hoc networking techniques.
\end{itemize}
\end{frame}

\subsection{Sensor networks communication architecture}
\begin{frame}
\frametitle{Sensor networks communication architecture}
\begin{itemize}
\item The sensor nodes are usually scattered in a sensor field as shown in the figure.
\begin{figure}
\begin{center}
  \includegraphics[width=0.68\textwidth]{l1}
\end{center}
\end{figure}
\end{itemize}
\end{frame}

\subsection{Design factors}
\begin{frame}
\frametitle{Design factors}
\begin{itemize}
\item<1-> Fault tolerance
\item<2-> Scalability
\item<3-> Production costs
\item<4-> Hardware constraints
\item<5-> Sensor network topology
\item<6-> Environment
\item<7-> Transmission media 
\item<8-> Power consumption
\end{itemize}
\end{frame}

\subsection{Fault tolerance}
\begin{frame}
\frametitle{Fault tolerance}
Fault tolerance is the ability to sustain sensor network functionalities without any interruption due to sensor node failures.
\begin{center}
$R_{k}(t)=e^{\lambda}k^{t}$
\end{center}
\end{frame}

\subsection{Scalability}
\begin{frame}
\frametitle{Scalability}
The density can range from few sensor nodes to few hundred sensor nodes in a region, which can be less than 10m in diameter.
\begin{center}
$\mu(R)\approx \frac{(N \cdot \pi R^{2})}{A}$
\end{center}
\end{frame}

\subsection{Power consumption}
\begin{frame}
\frametitle{Power consumption}
\begin{itemize}
\item Power consumption can be divided into three domains:
\begin{itemize}
\item Sensing
\item Communication
\item Data processing
\end{itemize}
\end{itemize}
\end{frame}

\subsection{Protocol stack}
\begin{frame}
\frametitle{Protocol stack}
The protocol stack used by the sink and sensor networks is given as following.
\begin{figure}
\begin{center}
  \includegraphics[width=0.4\textwidth]{l2}
\end{center}
\end{figure}
\end{frame}



\section{Overview of WSN Routing}
\subsection{Routing Techniques}
\begin{frame}
	\frametitle{Routing Techniques}
	\begin{itemize}
	\item<1-> Flat
	\item<2-> Hierarchical
	\item<3-> Location-Based Routing
	\end{itemize}
\end{frame}

\subsection{Routing protocols}
\begin{frame}
	\frametitle{Routing protocols}
	\begin{itemize}
	\item Multipath-Based
	\item Query-Based
	\item Negotiation-Based
	\item QoS-Based
	\item Coherent-Based
	\end{itemize}
\end{frame}

\subsection{Routing Challenges and Design Issues in WSNs}
\begin{frame}
	\frametitle{Routing Challenges and Design Issues in WSNs}
	\begin{itemize}
	\item Node Deployment
	\item Energy Consumption without Losing Accuracy
	\item Data Reporting Model
	\item Fault Tolerance
	\item Scalability
	\item Network Dynamics
	\item Transmission Media
	\item Coverage
	\item Quality of Service
	\end{itemize}
\end{frame}

\subsubsection{Node deployment}
\begin{frame}
\frametitle{Node deployment}
Node Deployment can be either \textbf{deterministic} or \textbf{randomized}.
\end{frame}

\subsubsection{Energy Consumption without Losing Accuracy}
\begin{frame}
\frametitle{Energy Consumption without Losing Accuracy}
For sensors, the power supply is limited, energy conserving forms of communication and computation are essential.
\end{frame}

\subsubsection{Data Reporting Model}
\begin{frame}
\frametitle{Data Reporting Model}
Data reporting can be categorized as either time-driven(continuous), event-driven, query-driven, and hybrid.
\end{frame}

\subsubsection{Fault Tolerance}
\begin{frame}
\frametitle{Fault Tolerance}
Some sensor nodes may fail or be blocked for many reasons. The failure of sensor nodes should not affect the overall task of the sensor network.
\end{frame}

\subsubsection{Scalability}
\begin{frame}
\frametitle{Scalability}
\begin{enumerate}
\item<1-> The number of sensor nodes may be in the order of hundreds or thousands, or more.
\item<2-> Sensor network routing protocols should be scalable enough to respond to events in the environment.
\end{enumerate}
\end{frame}

\subsubsection{Quality of Service}
\begin{frame}
\frametitle{Quality of Service}
In some applications, data should be delivered within a certain period of time from the moment it is sensed, otherwise the data will be useless.
\end{frame}

\section{OTC in WSNs}
\begin{frame}
\frametitle{Opportunity-based Topology Control in WSNs}
\begin{flushright}
------an effective method to improve\\
 the network’s energy efficiency\\
 \indent \\
 \textit{originated by Yunhuai Liu, ICDCS 2009,INFOCOM 2010}
\end{flushright}
\end{frame}

\subsection{Backgrounds in Topology Control}
\subsubsection{Traditional methods}
\begin{frame}
\frametitle{Backgrounds in Topology Control}
\begin{itemize}
\item \textbf{Traditional topology control approaches:}\\
\indent\\
based on the assumption that a pair of nodes is either \textit{connected} or \textit{disconnected}
\indent\\
\begin{flushright}
------\textit{Connectivity-based Topology Control}
\end{flushright}
\end{itemize}
\end{frame}

\subsubsection{Practical Situations}
\begin{frame}
\frametitle{Backgrounds in Topology Control}
\begin{itemize}
\item \textbf{Practical Situations:}\\
\indent\\
There is a transitional region that contains intermittently connected links, or \textit{lossy links}.\\
\indent\\
Literatures shows that lossy links actually accounts a major part of wireless networks, in some particular cases, more than 90\%.
\end{itemize}
\end{frame}

\subsection{Opportunity-based Topology Control}
\subsubsection{Benefits}
\begin{frame}
\frametitle{Opportunity-based Topology Control}
\begin{itemize}
\item \textbf{Distinguish from traditional model: application of lossy links}\\
\indent\\
\item \textbf{Benefits: Energy-efficient}\\
Lossy links allow transmitters to reach more nodes when succeed. By taking the successful lossy links, transmission cost can be largely reduced.
\end{itemize}
\end{frame}

\subsubsection{Figure Example 3}
\begin{frame}
\frametitle{Introduction to Reachability}
\begin{figure}
\begin{center}
  \includegraphics[width=0.80\textwidth]{o3}
\end{center}
\end{figure}
\end{frame}

\subsubsection{Shortcuts and Challenges}
\begin{frame}
\frametitle{Opportunity-based Topology Control}
\begin{itemize}
\item \textbf{Shortcuts: Unreliability of Links}\\
Since links are not completely ensured of connection, a given transmission may not be able to reach every required nodes.\\
\end{itemize}
\end{frame}

\subsubsection{Figure Example 1}
\begin{frame}
\frametitle{Opportunity-based Topology Control}
\begin{figure}
\begin{center}
  \includegraphics[width=0.80\textwidth]{o1}
\end{center}
\end{figure}
\end{frame}

\subsection{Introduction to Reachability}
\subsubsection{Node Reachability}
\begin{frame}
\frametitle{Introduction to Reachability}
\begin{itemize}
\item \textbf{Definition} \textit{Node Reachability} $\lambda_{G}(v)$:\\
\indent\\
Given a node $v$ in a network $G$, $\lambda_{G}(v)$ is the probability that the sink can reach $v$ by a broadcast.
\end{itemize}
\end{frame}

\subsubsection{Network Reachability}
\begin{frame}
\frametitle{Introduction to Reachability}
\begin{itemize}
\item \textbf{Definition} \textit{Network Reachability} $\lambda(G)$:\\
\indent\\
Given a network $G(V,E)$, $\lambda(G)$ is defined as the ratio between the expected
number of reached nodes by the sink using a sink initialized
broadcast, and the total number of nodes in $V$ \\
\indent\\
i.e.  $\lambda(G)=\frac{E|V_{r}|}{|V|}=\frac{\Sigma_{v\in V} \lambda_{G}(v)}{|V|}$
\end{itemize}
\end{frame}

\subsection{Problem Statement}
\subsubsection{Problem Statement 1}
\begin{frame}
\frametitle{Problem Statement}
\begin{itemize}
\item Given a directed graph $G(V,E)$ and a threshold $\lambda_{TH}\in[0,1]$, assuming $\lambda(G)\geq \lambda_{TH}$, the problem is to find a sub-network $G_{R}(V,E_{R}) \subseteq G$ such that:
\begin{align*}
\textrm{minmize} &: \varepsilon(G_{R})\\
\textrm{subject to} &: \lambda(G_{R})\geq \lambda_{TH}
\end{align*}
\end{itemize}
\end{frame}

\subsubsection{Problem Statement 2}
\begin{frame}
\frametitle{Problem Statement}
\begin{itemize}
\item
\textbf{Key issue}:\\
\indent\\
\indent Compute the node reachability\\
\indent\\
\indent Compute the network reachability\\
\indent\\
\item
\begin{flushright}
------Both NP-hard problem
\end{flushright}
\end{itemize}
\end{frame}

\subsection{Algorithm Design}
\subsubsection{properties}
\begin{frame}
\frametitle{Algorithm Design: CONREAP}
\begin{itemize}
\item Only simple combinations of series and parallel topologies can be computed in polynomial time.\\
\begin{flushright}
------series-parallel networks
\end{flushright}
\indent\\
\indent\\
\item Link-disjoint trees belongs to the series-parallel structure.
\end{itemize}
\end{frame}

\subsubsection{steps}
\begin{frame}
\frametitle{Algorithm Design: CONREAP}
\begin{enumerate}
\item<1-> Measure link reliability $\lambda(u,v) $ by sending “hello” message to all neighbor nodes $u$
\item<2-> Receive tree information $\lambda_{T_{i}}(u)$ from all neighbor nodes $u$
\item<3-> For every tree $T_{i}$, v chooses $u_{i}$ to maximaze its reachability and update $\lambda_{T_{i}}(v)=\lambda_{T_{i}}(u_{i})\cdot \lambda_{\{u_{i},v\}}$
\item<4-> Select a tree to maximize $\lambda_{T_{i}}(v)$
\item<5-> Include the chosen links $\{u_{i},v \}$ in to the topology
\item<6-> Calculate new node reachability according to $\tilde{\lambda}_{G_{R}}(v)=1-\Pi^{K}_{i=1}(1-\lambda_{T_{i}}(v))$
\item<7-> Broadcast and exchange updated reachability and topology information with other nodes
\item<8-> If the quality is still below the threshold, repeat from 2
\end{enumerate}
\end{frame}

\subsection{Performance Evaluation}
\subsubsection{result 1}
\begin{frame}
\frametitle{Performance Evaluation}
\begin{figure}
\begin{center}
  \includegraphics[width=0.80\textwidth]{o4}
\end{center}
\end{figure}
\end{frame}

\section{Random Walks on Digraphs}
\subsection{Directed Graphs}
\begin{frame}
\frametitle{Directed Graphs}
\begin{itemize}
\item<1->
\begin{center}
Routing protocol\\
packet forwarding
\end{center}
\indent \\

\item<2->
\begin{center}
A wireless network\\ 
A (weighted) directed graph(A digraph)
\end{center}
\begin{flushright}
\item<3->  Yanhua~Li, Zhi-Li~Zhang INFOCOM~2010
\end{flushright}
\end{itemize}
\end{frame}

\begin{frame}
\frametitle{An Example Wireless Topology}
\begin{figure}
\begin{center}
  \includegraphics[width=0.88\textwidth]{r0}
\end{center}
\end{figure}
Matrix $A=[a_{ij}]$ \\
($a_{ij}$ represents the the link probability from i to j)
\end{frame}

\subsection{Three Wireless routing paradigms}
\subsubsection{Best Path Routing}
\begin{frame}
\frametitle{Best Path Routing}
\begin{figure}
\begin{center}
  \includegraphics[width=0.56\textwidth]{r1}
\end{center}
\end{figure}
\begin{center}
Selects a optimistic sequence of transmission.
\end{center}
The transition probability matrix $P_{R}=[p_{ij}]$:
\begin{equation}
p_{ij} = \left\{ \begin{array}{ll}
a_{i,i+1} & \textrm{if $j=i+1,i=0,...,m$}\\
1-a_{i,i+1} & \textrm{if $j=i,i=0,...,m$}\\
1 & \textrm{if $j=i,i=m+1$}\\
0 & \textrm{otherwise.}
\end{array} \right.
\end{equation}
\end{frame}

\subsubsection{Opportunistic Routing}
\begin{frame}
\frametitle{Opportunistic Routing}
\begin{figure}
\begin{center}
  \includegraphics[width=0.56\textwidth]{r2}
\end{center}
\end{figure}
\begin{center}
Considers multiple paths in a subgraph,\\
which determines the "best" forwarder list.
\end{center}
The transition probability matrix $P_{FL}=[p_{ij}]$:
\begin{equation}
p_{ij} = \left\{ \begin{array}{ll}
a_{ij}\prod_{k>j}(1-a_{ik}) & \textrm{if $0\le i<j\le m+1$}\\
\prod_{k>i}(1-a_{ik}) & \textrm{if $j=i,0\le i \le m$}\\
1 & \textrm{if $j=i,i=m+1$}\\
0 & \textrm{otherwise.}
\end{array} \right.
\end{equation}
\end{frame}

\subsubsection{Stateless Routing}
\begin{frame}
\frametitle{Stateless Routing}
\begin{figure}
\begin{center}
  \includegraphics[width=0.56\textwidth]{r3}
\end{center}
\end{figure}
\begin{center}
Every node has a probability to taking part in the forwarding.
\end{center}
The transition probability matrix $P_{G}=[p_{ij}]$:
\begin{equation}
p_{ij} = \left\{ \begin{array}{ll}
\frac{a_{ij}}{\sum_{k}a_{ik}}(1-\prod_{k}(1-a_{ik})) & \textrm{if $i\not= j$}\\
\prod_{k}(1-a_{ik}) & \textrm{if $i=j$}
\end{array} \right.
\end{equation}
\end{frame}

\begin{frame}
\frametitle{Three Wireless routing paradigms}
\begin{itemize}
\item<1-> Best Path Routing:
\begin{itemize}
\item Static wireless environment
\item Stable and reliable wireless channels
\end{itemize}
\item<2-> Opportunistic Routing:
\begin{itemize}
\item Less reliable wireless channels
\item Frequently varying conditions
\end{itemize}
\item<3-> Stateless Routing
\begin{itemize}
\item Unstable wireless network environment
\end{itemize}
\end{itemize}
\end{frame}

\subsection{Transmission Cost}
\begin{frame}
\frametitle{Transmission Cost}
\begin{itemize}
\item<1-> Hitting Time $H_{ij}$: the (expected) number of transitions for a random walker that starts from node $i$ to first reach node $j$.
\begin{equation}
H_{ij} = \left\{ \begin{array}{ll}
1+ \sum_{k=1}^{n}p_{ik}H_{kj} & \textrm{if $i\not= j$}\\
0 & \textrm{if $i=j$}
\end{array} \right.
\end{equation}
\item<2-> Hitting Cost $H_{ij}^{s}$: the (expected) totle cost (or "delay") incurred by a random walk that starts at one node $i$ to first reach node $j$.
\begin{equation}
H_{ij}^{s} = \left\{ \begin{array}{ll}
\sum_{k=1}^{n}p_{ik}(T_{ik}+H_{kj}^{s}) = s_{i}+ \sum_{k=1}^{n} p_{ik} H_{kj}^{s} & \textrm{if $i\not= j$}\\
0 & \textrm{if $i=j$}
\end{array} \right.
\end{equation}
\end{itemize}
\end{frame}

\end{document}