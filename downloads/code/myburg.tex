\documentclass[11pt,a4paper]{ctexart}
\usepackage[top=1in,bottom=1in,left=1.25in,right=1.25in]{geometry}
\usepackage[bf,small,pagestyles]{titlesec}
\usepackage[dvips]{graphicx}
\usepackage{float}
\usepackage{amssymb}
\usepackage{indentfirst}
\usepackage{listings}
\lstset{
	escapechar=`,
	language=Matlab, tabsize=2, keepspaces=true,
    xleftmargin=2em, xrightmargin=2em, aboveskip=1em,
    frame=none,   
    basicstyle=\scriptsize,
    showspaces=false,
    flexiblecolumns=true,
    breaklines=true, breakautoindent=true,breakindent=4em,
}
\def\pgfsysdriver{pgfsys-dvipdfmx.def}
\usepackage{tikz}
\pgfsetxvec{\pgfpoint{10pt}{0}}
\pgfsetyvec{\pgfpoint{0}{10pt}}
\tikzset{box/.style={rectangle, rounded corners=6pt,minimum width=50pt, minimum height=20pt, inner sep=6pt,draw=black,thick, fill=white}}
\pagestyle{plain}

\begin{document}
\title{\textbf{数字语音处理实验报告:实现Burg算法}}
\author{王帆  5070309862  F0703034}
\date{\today}
\maketitle

\section{实验目的}
\begin{enumerate}
\item{理解语音信号的线性预测方法}
\item{理解Burg算法}
\end{enumerate}

\section{实验内容}
编写一个程序,实现Burg算法。\\

由已知数据计算预测系数的方法有三:自相关法或Yule-Walker法、协方差法、Burg法。
\begin{itemize}
\item{自相关法} 相应的Yule-Walker方程是Toeplitz方程,可用Levinson算法高效的求解,理论上所得预测误差滤波器是稳定的。但在计算预测误差时,数据段两端都要添加零取样值,即相当于加了以数据窗,造成谱估计失真。另外,当预测系数量化时,可能造成实际系统的不稳定。
\item{协方差法} 重新定义时间平均最小均方准则:$\varepsilon' = \frac{1}{N} \sum^{N-1}_{n=p} e^{2}_{p}(n) = min$。数据两端不需要添加零取样值。其方法计算出来的预测系数有可能造成预测误差滤波器的不稳定。
\item{Burg法} 重新定义时间平均最小均方准则:$\varepsilon = \sum^{N-1}_{n=p}[(e^{+}_{p}(n))^{2}+(e^{-}_{p}(n))^{2}] = min$。即要求前向预测误差和后向预测误差的平方和(在$n=p$到$n=N-1$时间间隔内)最小。Burg算法保证了滤波器总是稳定的(最小相位的),即要求$|k_{p}|<1$。
\end{itemize}

Burg算法的计算步骤:
\begin{enumerate}
\item 初始化
\begin{displaymath}
e^{+}_{0}(n)=e^{-}_{0}(n)=x(n), \qquad 0\leqslant n \leqslant N-1 
\end{displaymath}
\begin{displaymath}
\varepsilon_{0}=\frac{1}{N}\sum^{N-1}_{n=0} x^{2}(n)
\end{displaymath}

\item 计算下列各量:
\begin{displaymath}
A_{p-1}(z), \varepsilon_{p-1}, e^{+}_{p-1}(n),  e^{-}_{p-1}(n), \qquad p-1\leqslant n \leqslant N-1
\end{displaymath}


\item 计算反射系数
\begin{displaymath}
k_{p}=\frac{\displaystyle 2\sum^{N-1}_{n=p} e^{+}_{p-1}(n)e^{-}_{p-1}(n-1)}{\displaystyle \sum^{N-1}_{n=p}[(e^{+}_{p-1}(n))^{2}+(e^{-}_{p-1}(n-1))^{2}]}
\end{displaymath}

\item 计算$A_{p}(z)$
\begin{displaymath}
\left[ \begin{array}{c}
1 \\
a_{p,1} \\
\vdots \\
a_{p,p-1} \\
a_{p,p} \end{array} \right]
=
\left[ \begin{array}{c}
1 \\
a_{p-1,1} \\
\vdots \\
a_{p-1,p-1} \\
0 \end{array} \right]
- k_{p}
\left[ \begin{array}{c}
0 \\
a_{p-1,p-1} \\
\vdots \\
a_{p-1,1} \\
1 \end{array} \right]
\end{displaymath}
即:$a_{p,j}=a_{p-1,j}-k_{p}a_{p-1,p-j},\qquad j\leqslant p-1$。

\item 计算$e^{+}_{p}(n)$和$e^{-}_{p}(n)$
\begin{displaymath}
\left\{ \begin{array}{ll}
e^{+}_{p}(n) & =e^{+}_{p-1}(n)-k_{p}e^{-}_{p-1}(n-1),\\
e^{-}_{p}(n) & =e^{-}_{p-1}(n-1)-k_{p}e^{+}_{p-1}(n),
\end{array}
\qquad p \leqslant n \leqslant N-1 \right.
\end{displaymath}

\item 更新均方误差(实验中没有做处理)
\begin{displaymath}
\varepsilon_{p} = (1-k^{2}_{p})\varepsilon_{p-1}
\end{displaymath}

\item 返回步骤2。
.\end{enumerate}

\section{实验代码}
\begin{lstlisting}
X=[5 0 7 0 3 0 9 8 6 2];%`样本值`
p=4;%`阶数`
N=length(X);%`样本数`
e0(1,:)=X;%`前向预测和后向预测的初始值`
e1(1,:)=X;%`前向预测和后向预测的初始值`
a=zeros(p+1,p+1);
a(:,1)=1;%`预测系数矩阵初始化`
k=[];%`反射系数矩阵初始化`
for i=2:p+1
    num=2*(e0(i-1,[i:N])*e1(i-1,[i-1:N-1])');
    den=e0(i-1,[i:N])*e0(i-1,[i:N])'+e1(i-1,[i-1:N-1])*e1(i-1,[i-1:N-1])';
    k(i)=num/den;%`计算反射系数`

    for j=2:i-1
        a(i,j)=a(i-1,j)-k(i)*a(i-1,i-j+1);
    end
    a(i,i)=-k(i);%`计算预测系数`

    for m=2:N
        e0(i,m)=e0(i-1,m)-k(i)*e1(i-1,m-1);
        e1(i,m)=e1(i-1,m-1)-k(i)*e0(i-1,m);
    end%`计算前向预测和后向预测`
end
k%`查看稳定性`
This_arburg=a
Matlab_arburg=arburg(X,p)%`对比试验结果`
\end{lstlisting}

\section{实验结果}
计算4阶的实验结果:
\begin{lstlisting}
k =

         0    0.5207    0.5121   -0.3052    0.6825


This_arburg =

    1.0000         0         0         0         0
    1.0000   -0.5207         0         0         0
    1.0000   -0.2541   -0.5121         0         0
    1.0000   -0.4104   -0.5896    0.3052         0
    1.0000   -0.6187   -0.1872    0.5853   -0.6825


Matlab_arburg =

    1.0000   -0.6187   -0.1872    0.5853   -0.6825
\end{lstlisting}

计算5阶的实验结果:
\begin{lstlisting}
k =

         0    0.5207    0.5121   -0.3052    0.6825    0.1517


This_arburg =

    1.0000         0         0         0         0         0
    1.0000   -0.5207         0         0         0         0
    1.0000   -0.2541   -0.5121         0         0         0
    1.0000   -0.4104   -0.5896    0.3052         0         0
    1.0000   -0.6187   -0.1872    0.5853   -0.6825         0
    1.0000   -0.5151   -0.2760    0.6137   -0.5887   -0.1517


Matlab_arburg =

    1.0000   -0.5151   -0.2760    0.6137   -0.5887   -0.1517
\end{lstlisting}
\section{心得体会}
这次试验主要是编写一个程序,实现Burg算法。实验结果如上所示,和Matlab自带的arburg函数的计算结果吻合。同时反射系数满足$|k_{p}|<1$,即符合Burg算法的要求。实验内容难度不大,按照步骤实现程序就能完成。

\end{document}
